\documentclass[bwprint, withoutpreface]{cumcmthesis}

\usepackage{extarrows}
\usepackage{authblk}
\usepackage{tikz}
\usepackage{xcolor}
\usetikzlibrary{intersections, calc, bending, decorations.markings, arrows, shapes, positioning, decorations.pathreplacing, shadows, arrows.meta}
\newcommand*{\dif}{\mathop{}\!\mathrm{d}}

\title{实变第四章总结}

\begin{document}
\maketitle
\noindent Author: Tony Xiang

\noindent Full Document can be acquired here: 

\noindent https://github.com/T0nyX1ang/RealAnaly-Documents/blob/master/Chapter\%204/Chapter4.pdf

\noindent Full Source code can be downloaded here:

\noindent https://github.com/T0nyX1ang/RealAnaly-Documents/blob/master/Chapter\%204/Chapter4.tex

\section{Lebesgue积分的定义}
分三步定义:
\subsection{非负简单函数的积分:}
\indent 定义:设$f(x) = \sum_{i = 1}^{k}{a_i\chi_{A_i}(x)}$是$E$上的非负简单函数,其中$\{A_1, A_2, \cdots, A_k\}$是$E$的一个可测分割,$a_i \in \mathbb{R}^n, i = 1, 2, \cdots, k$.定义$f$在$E$上的积分为:
\begin{equation*}
	\int_E{f} \dif x = \sum_{i = 1}^{k}{a_i m(A_i)}
\end{equation*}
\begin{equation*}
	0 \leqslant \int_E{f} \dif x \leqslant \infty, \mbox{若} \int_E{f} \dif x < \infty, \mbox{则}f \in L(E).
\end{equation*}

注意,该积分的值不依赖$f$的表达式的选取.证明可取另一个可测分割$\{B_i\}$,则有$m(A_i) = \sum_{j = 1}^{l}{m(A_i \cap B_j)}, (i = 1, 2, \cdots, k), m(B_j) = \sum_{i = 1}^{k}{m(A_i \cap B_j)}, (j = 1, 2, \cdots, l)$, $A_i \cap B_j \neq \emptyset, a_i = b_j$,可得 \[ \sum_{i = 1}^{k} a_i m(A_i) = \sum_{i = 1}^{k}{\sum_{j = 1}^{l}{a_i m(A_i \cap B_j)}} = \sum_{j = 1}^{l}{\sum_{i = 1}^{k}{b_j m(A_i \cap B_j)}} = \sum_{j = 1}^{l}{b_j m(B_j)} \]

注意,$\int_{[a, b]} f \dif x$为函数$y= f(x)$的下方图形的积分.

性质(Part 1, $f, g$为非负简单函数):
\begin{itemize}[itemindent=2em]
	\item(特征函数的性质) \[ A \subset E \in \mathcal{M}(\mathbb{R}^n), \int_E \chi_A(x) \dif x = m(A). \]特别地,\[ \int_E \chi_E(x) \dif x = m(E). \]
	\item(数乘的性质) \[ \int_E cf \dif x = c \int_E f \dif x \quad (c \geqslant 0). \]
	\item(加法的性质) \[ \int_E f + g \dif x = \int_E f \dif x + \int_E g \dif x. \]
	\item(单调性)\[ f \leqslant g \mathop{} \! \mathrm{a.e.} \Leftarrow \int_E f \dif x \leqslant \int_E g \dif x. \]
\end{itemize}

第一、二条性质的证明可以直接利用定义,第三条性质可以将两个函数的可测分割记为一样的,然后使用第一条性质,第四条性质有$m(E_i) > 0, a_i \leqslant b_i$,然后使用定义证明.

注意,Lebesgue积分充分利用了零测集对于积分值没有影响的特点,所以在表示不等关系的时候,一般使用“几乎处处”的条件就足够了.

\subsection{非负可测函数的积分}
\indent 引理(证明略):设$\{f_n\}$是$E$上单调递增的非负简单函数列,
\begin{itemize}[itemindent=2em]
	\item 若$g$是$E$上的非负简单函数,且$\lim_{n \to \infty} f_n(x) \geqslant g(x)$,则\[ \lim_{n \to \infty}{\int_E f_n \dif x \geqslant \int_E g \dif x} \]
	\item 若$\lim_{n \to \infty}{f_n(x) = f(x), x \in E}$,则 \[ \lim_{n \to \infty}{\int_E f_n \dif x} = \sup{\{ \int_E g \dif x: g \in S^+{E}, g \geqslant f \}} \],S+(E)表示$E$上的非负简单函数的全体.
\end{itemize}

定义:设$f$是$E$上的非负可测函数,$f_n$是$E$上的非负可测函数列,且$f_n \nearrow f$,定义$f$在$E$上的积分为:
\begin{equation*}
	\int_E f \dif x = \lim_{n \to \infty} \int_E f_n \dif x 
\end{equation*}
\begin{equation*}
	0 \leqslant \int_E{f} \dif x \leqslant \infty, \mbox{若} \int_E{f} \dif x < \infty, \mbox{则}f \in L(E).
\end{equation*}

同样,$f$在$E$上的积分值不依赖与$\{f_n\}$的选取.

性质(Part 2, $f, g$为非负可测函数):
\begin{itemize}[itemindent=2em]
	\item(数乘的性质) \[ \int_E cf \dif x = c \int_E f \dif x \quad (c \geqslant 0). \]
	\item(加法的性质) \[ \int_E f + g \dif x = \int_E f \dif x + \int_E g \dif x. \]
	\item(单调性)\[ f \leqslant g \mathop{} \! \mathrm{a.e.} \Rightarrow \int_E f \dif x \leqslant \int_E g \dif x. \]
\end{itemize}

证明:第一条性质显然成立,第二条性质可以用两个非负函数列分别逼近$f$和$g$,第三条性质可以选取适当的$\{f_n\}, \{g_n\}, f_n \leqslant g_n, n \geqslant 1$,且有$f_n \nearrow f, g_n \nearrow g$,则有 \[ \int_E f \dif x = \lim_{n \to \infty}{\int_E f_n \dif x} \leqslant \lim_{n \to \infty}{\int_E g_n \dif x} = \int_E g \dif x. \]

\subsection{一般可测函数的积分}
\indent 定义:设$f$是$E$上的可测函数,若$\int_E f^+ \dif x, \int_E f^- \dif x$至少有一个是有限值,则$f$在$E$上的积分存在,且定义为:
\begin{equation*}
	\int_E f \dif x = \int_E f^+ \dif x - \int_E F^- \dif x.
\end{equation*}

若$\int_E f^+ \dif x, \int_E f^- \dif x$都是有限值,则$f \in L(E)$. 将$[a, b]$区间上的Lebesgue积分记为$\int_{a}^{b}{f \dif x}$.

基本的可积条件(Part 3, $f, g$为一般函数):
\begin{itemize}[itemindent=2em]
	\item $g \in L(E), f(x) \leqslant g(x), \mathop{} \! \mathrm{a.e.} x \in E \mbox{或} f(x) \geqslant g(x), \mathop{} \! \mathrm{a.e.} x \in E$,$f$在$E$上的积分存在.
	\item $g \in L(E), |f(x)| \leqslant g(x), \mathop{} \! \mathrm{a.e.} x \in E$,则$f \in L(E)$.
	\item $f \in L(E) \Leftrightarrow |f| \in L(E)$.
	\item $m(E) < \infty, \exists M, |f| \leqslant M, f \in L(E)$.
\end{itemize}

证明:第一条性质有$f^+(x) \leqslant g^+(x) \mathop{} \! \mathrm{a.e.} x \in E \mbox{或} f^-(x) \leqslant g^-(x) \mathop{} \! \mathrm{a.e.} x \in E$.,然后用积分的单调性可得$\int_E f^+ \dif x, \int_E f^- \dif x$至少有一个是有限值.第二条性质兼具第一条性质的条件,故有可积性.第三条性质$|f| = f^+ + f^-$,可得$\int_E |f| \dif x \mbox{是有限值} \Leftrightarrow \int_E f^+ \dif x, \int_E f^- \dif x \mbox{都是有限值}$.第四条性质将$g(x)=M, x \in E$作为控制函数即可证明.

\textbf{一个通用的证明过程:从非负简单函数到非负可测函数再到一般函数的证明方法.}

\begin{center}
\begin{tikzpicture}
[
basic/.style={
% The shape:
rectangle,
% The size:
minimum height=6mm,
% The border:
very thick,
draw=red!50!black!50, % 50% red and 50% black,
% and that mixed with 50% white
% The filling:
top color=white, % a shading that is white at the top...
bottom color=red!50!black!20, % and something else at the bottom
},
proof/.style={
% The shape:
rectangle,
% The size:
minimum height=6mm,
% The border:
very thick,
draw=gray!50!black!50, % 50% red and 50% black,
% and that mixed with 50% white
% The filling:
top color=white, % a shading that is white at the top...
bottom color=gray!50!black!20, % and something else at the bottom
},
adapter/.style={
% The shape:
rectangle,
% The size:
minimum width=3cm,
minimum height=6mm,
% The border:
very thick,
draw=green!50!black!50, % 50% red and 50% black,
% and that mixed with 50% white
% The filling:
top color=white, % a shading that is white at the top...
bottom color=green!50!black!20, % and something else at the bottom
},
decorator/.style={
% The shape:
rectangle,
% The size:
minimum width=1cm,
minimum height=6mm,
% The border:
very thick,
draw=blue!50!black!50, % 50% red and 50% black,
% and that mixed with 50% white
% The filling:
top color=white, % a shading that is white at the top...
bottom color=blue!50!black!20, % and something else at the bottom
}
]
    \node[below, align=justify, basic] (a) at (2.5, 0) {非负简单函数};
    \node[below, align=justify, decorator] (c) at (12.5, 0) {一般函数};
    \node[below, align=justify, adapter] (b) at (7.5, 0) {非负可测函数};
    \node[below, align=justify, proof] (d1) at (-2, -1.5) {定义};
    \node[below, align=justify, proof] (d2) at (2, -1.5) {单增的非负简单函数};
    \node[below, align=justify, proof] (d3) at (8, -1.5) {正部与负部};
    \draw[-{Stealth[black]}, line width=1pt] (a.east) -- (b.west);
    \draw[-{Stealth[black]}, line width=1pt] (b.east) -- (c.west);
    \draw[-{Stealth[black]}, line width=1pt, densely dashed] (d1.east) -- (a.west);
    \draw[-{Stealth[black]}, line width=1pt, densely dashed] (d2.east) -- (b.west);
    \draw[-{Stealth[black]}, line width=1pt, densely dashed] (d3.east) -- (c.west);
\end{tikzpicture}
\end{center}


\textbf{积分区域的变化与特征函数的关系:设$f$在$E$上的积分存在,$A$是$E$的可测子集,则$f$在$A$上的积分存在,且\[ \int_A f \dif x = \int_E f \chi_A \dif x \]同样有$f \in L(E) \Rightarrow f \in L(A).$}(证明方法为以上的分三个步骤的方法,第一步按定义证明,将$E$上的可测分割限定到$A$上;第二步取$f_n \chi_A \nearrow f \chi_A$;第三步利用正部和负部处理一般函数)

以上的定理常常用在证明\textbf{多个积分区域不统一}的问题中.

积分的平移不变性:$f(x) \in L(\mathbb{R}^n), h \in \mathbb{R}^n, \mbox{则} f(x + h) \in L(\mathbb{R}^n)$,且有
\begin{equation*}
	\int_{\mathbb{R}^n} f(x + h) \dif x = \int_{\mathbb{R}^n} f(x) \dif x.
\end{equation*}

同样可以分三步证明,其中第一步用到测度的平移不变性.后面两步类似.

\section{积分的初等性质}
\subsection{线性性}
若$f,g \in L(E)$,则$cf, f + g \in L(E)$,
\begin{equation*}
	\int_E cf \dif x = c \int_E f \dif x
\end{equation*}
\begin{equation*}
	\int_E (f + g) \dif x = \int_E f \dif x + \int_E g \dif x
\end{equation*}

证明:$f \in L(E) \Rightarrow |f| \in L(E)$,则\[ \int_E |cf| \dif x = \int_E |c||f| \dif x = |c| \int_E |f| \dif x < \infty. \]

故$|cf| \in L(E)$,$cf \in L(E)$,类似有$|f + g| \leqslant |f| + |g| \Rightarrow f + g \in L(E).$利用

\begin{equation*}
	(cf)^+ =
	\begin{cases}
		cf^+ \quad c \geqslant 0 \\
		-cf^- \quad c \geqslant 0
	\end{cases}
\end{equation*}
\begin{equation*}
	(cf)^- =
	\begin{cases}
		cf^- \quad c \geqslant 0 \\
		-cf^+ \quad c \geqslant 0
	\end{cases}
\end{equation*}
\begin{equation*}
	\int_E cf \dif x = \int_E cf^+ \dif x - \int_E cf^- \dif x = c \int_E f^+ \dif x - c \int_E f^- \dif x = c \int_E f \dif x.
\end{equation*}
\begin{equation*}
	(f + g)^+ - (f + g)^- = f + g = f^+ - f^- + g^+ - g^- \Rightarrow (f + g)^+ + f^- + g^- = f^+ + g^+ + (f + g)^-
\end{equation*}
\begin{align*}
	\int_E (f + g) \dif x & = \int_E (f + g)^+ \dif x - \int_E (f + g)^- \dif x \\
						  & = \int_E f^+ \dif x - \int_E f^- \dif x + \int_E g^+ \dif x - \int_E g^- \dif x \\  
						  & = \int_E f \dif x + \int_E g \dif x
\end{align*}

\subsection{对积分域的有限可加性}
设$f \in L(E), A_1 \cap A_2 = \emptyset, E = A_1 \cup A_2,$则 \[	\int_E f \dif x = \int_{A_1} f \dif x + \int_{A_2} f \dif x \]

证明:$f \in L(A_1), f \in L(A_2)$,得到$f\chi_{A_1}, F\chi_{A_2} \in L(E)$,有 \[ \int_{A_1} f \dif x + \int_{A_2} f \dif x = \int_E f \chi_{A_1} \dif x + \int_E f \chi_{A_2} = \int_E (f \chi_{A_1} + f \chi_{A_2}) \dif x = \int_E f \dif x. \]

同样可证有限情况:$f \in L(E)$,$\{A_1, A_2, \cdots, A_n\}$为$E$的一个可测分割,则有:
\begin{equation*}
	\int_E f \dif x = \sum_{k = 1}^{n}{\inf_{A_k}{f \dif x}}
\end{equation*}

\subsection{积分的单调性与不等关系}
设$f, g$在$E$上的积分存在,则:
\begin{itemize}[itemindent=2em]
	\item $f \leqslant g \Rightarrow \int_E f \dif x \leqslant \int_E g \dif x$.
	\item $f = g \mathop{} \! \mathrm{a.e.} \Rightarrow \int_E f \dif x = \int_E g \dif x$.
	\item $f \geqslant 0 \mathop{} \! \mathrm{a.e.} A, B \subset E, A \subset B \Rightarrow \int_A f \dif x \leqslant \int_B f \dif x$.
	\item $f = 0 \mathop{} \! \mathrm{a.e.} \Rightarrow \int_E f \dif x = 0$.
	\item $m(E) = 0, \forall f(x), x \in E \Rightarrow \int_E f \dif x = 0$
	\item $f \in L(E), |\int_E f \dif x| \leqslant \int_E |f| \dif x$.
\end{itemize}

证明:(1)若在$E$上,$f \leqslant g \mathop{} \! \mathrm{a.e.}$,则$f^+ \leqslant g^+, f^- \geqslant g^- \mathop{} \! \mathrm{a.e.}$ 则
\begin{equation*}
	\int_E f^+ \dif x \leqslant \int_E g^+ \dif x, \int_E f^- \dif x \geqslant \int_E g^+ \dif x
\end{equation*}
\begin{equation*}
	\int_E f \dif x = \int_E f^+ \dif x - \int_E f^- \dif x \leqslant \int_E g^+ \dif x - \int_E g^- \dif x = \int_E g \dif x.
\end{equation*}

(2)由(1)的对偶性可得.

(3)由$f \chi_A(x) \leqslant f \chi_B(x), x \in E$使用(1)的结论即可.

(4)由(2)可得.

(5)利用$\int_E f = 0$推出$f = 0 \mathop{} \! \mathrm{a.e.}$由(4)可得.

(6)对$-|f| \leqslant f \leqslant |f|$积分即可.

\subsection{Lebesgue积分的原始定义}
设$m(E) < \infty$,$f$是$E$上可测函数,$c \leqslant f(x) < d,(x \in E), \forall n \in \mathbb{N}^+$,设$c = y_0 < y_1 < \cdots < y_n = d$为$[c, d]$的一个分割,令$\lambda = \max_{1 \leqslant i \leqslant n}{y_i - y_{i - 1}}$.则
\begin{equation*}
	\int_E f \dif x = \lim_{\lambda \to 0}{\sum_{i = 1}^{n}{y_{i - 1} \cdot m(E(y_{i - 1} \leqslant f < y))}}.
\end{equation*}

证明:$f$是有限测度集上的有界可测函数,则$f \in L(E)$,令$E_i = E(y_{i - 1} - y{i}), (i = 1, 2, \cdots, n)$,则$\{E_i\}$为$E$的一个可测分割,利用积分的单调性与对积分域的有限可加性,有
\begin{equation*}
	\sum_{i = 1}^{n}{y_{i - 1}} m(E_i) = \sum_{i = 1}^{n}{\int_{E_i} y_{i - 1} \dif x} \leqslant \sum_{i = 1}^{n}{\int_{E_i} f \dif x} = \int_E f \dif x.
\end{equation*}
\begin{equation*}
	\int_E f \dif x \leqslant \sum_{i = 1}^{n}{y_i m(E_i)}
\end{equation*}

所以有

\begin{align*}
	 0 & \leqslant \int_E f \dif x - \sum_{i = 1}^{n}{y_{i - 1}m(E_i)} \leqslant \sum_{i = 1}^{n}{y_i m(E_i)} - \sum_{i = 1}^{n}{y_{i - 1} m(E_i)} \\ 
	   & = \sum_{i = 1}^{n}{(y_i - y_{i - 1})} m(E_i) \leqslant \lambda m(E) \to 0.
\end{align*}

\begin{center}
	\begin{tikzpicture}
		\draw [-{Stealth[black]}, line width=1pt] (-1, 0) -- (10, 0);
		\draw [-{Stealth[black]}, line width=1pt] (0, -1) -- (0, 5);
		\draw [line width=1pt] (1, 1) .. controls (2, 3) and (3, 3) .. (4, 2) 
									  .. controls (5, 1) and (6, 1) .. (8, 4);
		\foreach \x in {1,...,9}
			\draw (0, \x/2) -- (10, \x/2);

		\foreach \y in {1,...,9}
			\draw (\y, 0) -- (\y, 5);

		\node[align=justify, below] at (5, 0) {Riemann积分的方向};
		\node[align=justify, left] at (0, 2.5) {Lebesgue积分的方向};
	\end{tikzpicture}
\end{center}

\subsection{Chebyshev不等式与应用}
\indent \textbf{Chebyshev不等式:}设$f$是$E$上的可测函数,则$\forall \lambda > 0, m(E(|f| \geqslant \lambda)) = \frac{1}{k} \int_E |f| \dif x$.

证明:$x \in E(|f| \geqslant \lambda)$,有$\frac{1}{\lambda}|f(x)| \geqslant 1$.
\begin{equation*}
	m(E(|f| \geqslant \lambda)) = \int_{E(|f| \geqslant \lambda)} 1 \dif x \leqslant \frac{1}{k}\int_{E(|f| \geqslant \lambda)} |f| \dif x \leqslant \frac{1}{\lambda} \int_E |f| \dif x
\end{equation*}

注意,有时候我们需要使用以下减弱的不等式:
\begin{equation*}
	m(E(|f| \geqslant \lambda)) \leqslant \frac{1}{k}\int_{E(|f| \geqslant \lambda)} |f| \dif x.
\end{equation*}

应用1:$f \in L(E) \Rightarrow f$在$E$上几乎处处有限.

$f \in L(E), |f| \in L(E)$,$A = E(|f| = \infty), A_k = E(|f| \geqslant k), k = 1, 2, \cdots$. 则$A \subset A_k$. 利用Chebyshev不等式得到:$0 \leqslant m(A) \leqslant m(A_k) \leqslant \frac{1}{k} \int_E |f| \dif x \to 0, (k \to \infty)$,$m(A) = 0$.

[另外也可以用测度的上连续性,在$n = 1$的情况下使用Chebyshev不等式.]

应用2:$f \geqslant 0 \mathop{} \! \mathrm{a.e.}, \int_E f \dif x \Rightarrow f = 0 \mathop{} \! \mathrm{a.e.}$

$f \geqslant 0 \mathop{} \! \mathrm{a.e.}$,故$m(E(f < 0)) = 0$.$A = E(f > 0), A_k = E(f > \frac{1}{k}), A = \bigcup_{k = 1}^{\infty}{A_k}$.利用Chebyshev不等式得到:$0 \leqslant m(A_k) \leqslant k \int_E f \dif x = 0$. $m(A_k) = 0$,由测度的次可列可加性$m(A) = 0$,则$f = 0 \mathop{} \! \mathrm{a.e.}$

[另外也可以用测度的下连续性.]

\subsection{积分的绝对连续性}
设$f \in L(E)$,则$\forall \varepsilon > 0, \exists \delta > 0, s.t. \quad A \subset E, m(A) < \delta, \int_A |f| \dif x < \varepsilon.$

用泛函的方式理解:

\begin{equation*}
	\lim_{m(A) \to 0}{\int_A |f| \dif x = 0}.
\end{equation*}

证明:设$f \in L(E)$,则$|f| \in L(E)$.$\{g_k\}$为非负简单函数列,$g_k \nearrow |f|$. 由积分定义,
\begin{equation*}
	\lim_{k \to \infty} \int_E g_k \dif x = \int_E |f| \dif x < \infty.
\end{equation*}

于是$\forall \varepsilon > 0, \exists k_0, s.t.$ \[ 0 \leqslant \int_E (|f| - g_{k_0}) \dif x = \int_E |f| \dif x - \int_E g_{k_0} \dif x < \frac{\varepsilon}{2} \]

令$0 \leqslant M = \max_{x \in E}{g_{k_0}(x)} < \infty$,不妨设$M > 0$($M = 0$时显然),取$\delta = \frac{\varepsilon}{2M}$,再取$A \subset E, m(A) < \delta$,则有

\begin{equation*}
	\int_E |f| \dif x = \int_A (|f| - g_{k_0}) \dif x + \int_A g_{k_0} \dif x < \frac{\varepsilon}{2} + \int_A M \dif x = \frac{\varepsilon}{2} + Mm(A) < \varepsilon
\end{equation*}

注意,复值可测函数的积分,除序关系以外与实值可测函数的性质一致.

\section{积分的极限定理}
\subsection{Levi单调收敛定理}

\textbf{积分符号与极限符号的运算顺序的交换.}

设$\{f_n\}$是$E$上的单调递增的非负可测函数列,$f$是$E$上的非负可测函数,且$f_n \to f \mathop{} \! \mathrm{a.e.}$,则
\begin{equation*}
	\lim_E f_n \dif x = \int_E f \dif x.
\end{equation*}

证明:不妨设$f_n(x) \to f(x)$处处成立(因为改变一个零测集上的函数值不影响函数的整体性质,所以可以定义不收敛处的函数值为0),由积分的单调性得到:
\begin{equation*}
	\int_E f_n \dif x \leqslant \int_E f_{n + 1} \dif x \leqslant \int_E f \dif x. (n \geqslant 1)
\end{equation*}

因此$\lim_{n \to \infty}{f_n}$存在,并且$\lim_{n \to \infty}{\int_E f_n \dif x} \leqslant \int_E f \dif x.$

反过来,设$\{g_n\}$是非负可测函数列,且$g_n \nearrow f$.

$\forall k \geqslant 1$,$\lim_{n \to \infty}{f_n(x)} = f(x) \geqslant g_k(x), x \in E$.

由于$\lim_{n \to \infty}{\int_E f_n \dif x \geqslant \int_E g_k \dif x}$.令$k \to \infty$,可得

\begin{equation*}
	\lim_{k \to \infty}{\int_E f_n \dif x} \geqslant \int_E f \dif x.
\end{equation*}

\textbf{推论1(逐项积分定理):}$\{f_n\}$是$E$上的非负可测函数列,则

\begin{equation*}
	\int_E \sum_{n = 1}^{\infty}{f_n} \dif x = \sum_{n = 1}^{\infty}{\inf_E f \dif x}.
\end{equation*}

\textbf{积分符号与求和符号的运算顺序的交换.}

证明:令$g_n(x) = \sum_{i = 1}^{n}{f_i(x)},(n \geqslant 1)$,$f(x) = \sum_{i = 1}^{\infty}{f_i(x)}$,则$g_n \geqslant 0$,$g_n \nearrow f$,$f$是可测的.利用Levi定理,数列的替换与取极限过程,可得:

\begin{equation*}
	\int_E \sum_{n = 1}^{\infty}{f_n \dif x} = \lim_{n \to \infty}{\int_E g_n \dif x} = \lim_{n \to \infty}{\sum_{i = 1}^{n}{f_i} \dif x} = \sum_{i = 1}^{\infty}\int_E f_i \dif x.	
\end{equation*}

推论2(积分对积分域的可列可加性):$f$在$E$上的积分存在,$\{E_n\}$为$E$的可测分割,则\[ \int_E f \dif x = \sum_{n = 1}^{\infty}{\int_{E_n} f \dif x}. \]

证明:
\begin{equation*}
	\int_E f^+ \dif x = \int_E \sum_{n = 1}^{\infty}{f^+ \chi_{E_n} \dif x} = \sum_{n = 1}^{\infty}{\int_E f^+ \chi_{E_n} \dif x} = \sum_{n = 1}^{\infty}{\int_{E_n} f^+ \dif x}.
\end{equation*}

对负部可证类似结论,由于$f$的积分存在,$\int_E f^+ \dif x, \int_E f^- \dif x $至少有一个是有限的,即原命题得证.

\subsection{Fatou引理}
设$\{f_n\}$是$E$上的非负可测函数列,则
\begin{equation*}
	\int_E \varliminf_{n \to \infty} f_n \dif x \leqslant \varliminf_{n \to \infty} \int_E f_n \dif x. 
\end{equation*}

\textbf{单增条件减弱,结论随之减弱为不等关系,同时极限的顺序交换变成下极限的顺序交换.(“入不敷出”)}

证明:对每个$n \geqslant 1$,令$g_n = \inf_{k \geqslant n}{f_k(x)}, (x \in E)$.则$\{g_n\} \nearrow$,且$0 \leqslant g_n \leqslant f_n, \lim_{n \to \infty}{g_n} = \varliminf_{n \to \infty}{f_n}$. 由Levi定理得到:
\begin{equation*}
	\int_E \varliminf_{n \to \infty}{f_n} \dif x = \lim_{n \to \infty}{\int_E g_n \dif x \leqslant \varliminf_{n \to \infty}{\int_E f_n \dif x}}.
\end{equation*}

推论1:(一致有界的可积性)$f, f_n(n \geqslant 1)$是$E$上的可测函数,$f_n \to f \mathop{} \! \mathrm{a.e.}$,若$\sup_{n \geqslant 1}{\int_E |f_n| \dif x} < \infty$,则$f \in L(E)$.

证明:利用Fatou引理:
\begin{equation*}
	\int_E |f| \dif x = \int_E \lim_{n \to \infty}{|f_n|} \dif x \leqslant \varliminf_{n \to \infty}{\int_E |f_n| \dif x} \leqslant \sup_{n \geqslant 1}{\int_E |f_n| \dif x} < \infty.
\end{equation*}

\subsection{控制收敛定理}
设$f, f_n(n \geqslant 1)$是$E$上的可测函数,且$\exists g \in L(E), s.t. \quad |f_n| \leqslant g \mathop{} \! \mathrm{a.e.} (n \geqslant 1)$. 若在$E$上$f_n \to f \mathop{} \! \mathrm{a.e.}$或$f_n \stackrel{m}{\longrightarrow} f$,则$f, f_n \in L(E)$,并且
\begin{equation*}
	\lim_{n \to \infty} \int_E f_n \dif x = \int_E f \dif x.
\end{equation*}

证明:由于在$E$上$|f_n| \leqslant g \mathop{} \! \mathrm{a.e.} (n \geqslant 1)$,因此当$f_n \to f \mathop{} \! \mathrm{a.e.}$ 或 $f_n \stackrel{m}{\longrightarrow} f$时有$|f| \geqslant g \mathop{} \! \mathrm{a.e.}$(依测度收敛要使用Risez定理)由于$g \in L(E)$,可知$f, f_n \in L(E)$. 由
\begin{equation*}
	|\int_E f_n \dif x - \int_E f \dif x| = |\int_E (f_n - f) \dif x| \leqslant \int_E |f_n - f| \dif x.
\end{equation*}

下面证一个更强的结论:
\begin{equation*}
	\lim_{n \to \infty}{\int_E |f_n - f| \dif x} = 0.
\end{equation*}

分两种情况证明.先考虑$f_n \to f \mathop{} \! \mathrm{a.e.}$的情形.\textbf{令$h_n = 2g - |f_n - f|$},则$h_n \geqslant 0 \mathop{} \! \mathrm{a.e.}$,则$h_n \geqslant 0 \mathop{} \! \mathrm{a.e.} (n \geqslant 1)$.对函数列$\{h_n\}$应用Fatou引理,可得:
\begin{align*}
	\int_E 2g \dif x & = \int_E \lim_{n \to \infty}{(2g - |f_n - f|) \dif x}  \leqslant \varliminf_{n \to \infty}{\int_E (2g - |f_n - f|) \dif x} \\
				     & = \int_E 2g \dif x - \varlimsup_{n \to \infty}{\int_E |f_n - f| \dif x}
\end{align*}

又有:
\begin{equation*}
	0 \leqslant \varliminf_{n \to \infty}{\int_E |f_n - f| \dif x} \leqslant \varlimsup_{n \to \infty}{\int_E |f_n - f| \dif x} \leqslant 0
\end{equation*}

故可得加强结论成立.

再考虑$f_n \stackrel{m}{\longrightarrow} f$的情形,用反证法.则$\exists \varepsilon > 0, \{f_{n_k}\} \subset \{f_n\}, s.t. \quad \int_E |f_{n_k} - f| \dif x \geqslant \varepsilon \quad (k \geqslant 1).$由Risez定理,$\exists \{f_{n_{k'}}\} \subset \{f_{n_k}\}, s.t. \quad f_{n_k'} \to f \mathop{} \! \mathrm{a.e.} \quad (k' \to \infty)$,由上一种情形可得:
\begin{equation*}
	\lim_{k' \to \infty}{\int_E |f_{n_k'} - f| \dif x} = 0.
\end{equation*}
这与假设矛盾,故加强结论成立.

注意:此处的$\{h_n\}$的构造思路是依据Fatou引理与非负构造的,为了使极限值得到夹逼,从而在反方向构造函数.

注:若
\begin{equation*}
	\lim_{k \to \infty}{\int_E |f_{n_k} - f| \dif x} = 0.
\end{equation*}
成立,则称$\{f_n\}$在$L^1$中收敛于$f$,(或称平均收敛于$f$).

\textbf{推论1:(有界收敛定理)}设$m(E) < \infty$,$f, f_n(n \geqslant 1)$是$E$上的可测函数,且$\exists M, s.t. \quad |f_n| \leqslant M \mathop{} \! \mathrm{a.e.} (n \geqslant 1)$. 若在$E$上$f_n \to f \mathop{} \! \mathrm{a.e.}$或$f_n \stackrel{m}{\longrightarrow} f$,则$f, f_n \in L(E)$,并且
\begin{equation*}
	\lim_{n \to \infty} \int_E f_n \dif x = \int_E f \dif x.
\end{equation*}

证明:令$g(x) = M$作为$f_n(x)$的控制函数,结合$m(E) < \infty$即可.

\textbf{推论2:(积分号下求导)}设$f$是定义在$D = [a, b] \times [c, d]$上的实值函数,使得$\forall y \in [c, d], f(x, y) \in L([a, b])$,$\forall (x, y) \in D, {f_y}'$存在,并且存在控制函数$g(x) \in L([a, b])$:
\begin{equation*}
	|{f_y}'(x, y)| \leqslant g(x), (x, y) \in D.
\end{equation*}

则函数$I(y) = \int_{a}^{b} f(x, y) \dif x$在$[c, d]$上可导,且
\begin{equation}
	\frac{\dif}{\dif y} \int_{a}^{b} f(x, y) \dif x = \int_{a}^{b} {f_y}' (x, y) \dif x.
\end{equation}

证明:设$y \in [c, d]$,任取数列$\{h_n\}$,使得$y + h_n \subset [c, d], h_n \to 0, h_n \neq 0$. 令
\begin{equation*}
	\varphi_n(x) = \frac{f(x, y + h_n) - f(x, y)}{h_n} \quad (x \in [a, b]).
\end{equation*}

则$\lim_{n \to \infty} \varphi_n(x) = {f_y}'(x, y) \quad (x \in [a, b])$.

由微分中值定理,$x \in [a, b], \forall n \geqslant 1$, 
\begin{equation*}
	|\varphi_n(x)| = |\frac{f(x, y + h_n) - f(x, y)}{h_n}| = |{f_y}'(x, y + \theta h_n)| \leqslant g(x) \quad (0 < \theta < 1).
\end{equation*}

对$\{\phi_n\}$使用控制收敛定理,可得:
\begin{align*}
	\lim_{n \to \infty}{\frac{I(y + h_n) - I(y)}{h_n}} & = \lim_{n \to \infty} \frac{1}{h_n}{\int_{a}^{b}{[f(x, y + h_n) - f(x, y)] \dif x}} \\
													   & = \lim_{n \to \infty} {\int_{a}^{b} \varphi_n(x) \dif x} = \int_{a}^{b}{f_y}'(x, y) \dif x.
\end{align*}

则积分在$y$处可导,原命题成立.

注意:这个问题的证明方法是一种重要的证明思路:\textbf{当遇到函数极限的问题时,可以考虑先将其转换为数列极限(归结原理),在数列的框架下使用控制收敛定理等工具,然后在将其转换回函数形式(归结原理).}

\section{Lebesgue积分与Riemann积分的关系}
\subsection{Riemann积分}
\begin{itemize}[itemindent=2em]
	\item 分割与单调加细:$[a, b]$是有界闭区间,由$[a, b]$上有限个点构成的序列$\{P_n\}$称为$[a, b]$的一个分割,若$a = x_0 < x_1 < \cdots < x_n = b$.如果$\{P_n\}$是$[a, b]$的一列分割,使得$P_n \subset P_{n - 1} (n \geqslant 1)$,则称$\{P_n\}$是单调加细的.
	\item 相关记号与说明.$f$是$[a, b]$上的有界实值函数,$P = {\{x_i\}}_{i = 0}^{n}$为一个分割,定义:
	\begin{align*}
		& \Delta x_i = x_i - x_{i - 1} \\
		& m_i = \inf_{f(x): \mathop{} \! x \in [x_{i - 1}, x_i]} \\
		& M_i = \sup_{f(x): \mathop{} \! x \in [x_{i - 1}, x_i]} \\
		& \lambda = \max_{1 \leqslant i \leqslant n}{\Delta x_i} \mbox{为分割$P$的细度}.
	\end{align*}
	\item 上积分与下积分:
	\begin{equation*}
		\underline{\int_{a}^{b}} f \dif x = \sup_{\forall P} \sum_{i = 1}^{n}{m_i \Delta x_i}, \overline{\int_{a}^{b}} f \dif x = \inf_{\forall P} \sum_{i = 1}^{n}{M_i \Delta x_i}.
	\end{equation*}
	\item Riemann可积的充要条件:\textbf{上积分等于下积分}.且若一个函数在$[a, b]$上Riemann可积,则:
	\begin{equation*}
		\int_{a}^{b} f \dif x = \underline{\int_{a}^{b}} f \dif x = \overline{\int_{a}^{b}} f \dif x.
	\end{equation*}
\end{itemize}

\subsection{正常积分的关系}
\textbf{Step 1:构造Riemann积分与Lebesgue积分的桥梁.}
\begin{equation*}
	\underline{\int_{a}^{b}} f \dif x = \lim_{n \to \infty}{\sum_{i = 1}^{k_n}{m_i^{(n)}}}, \overline{\int_{a}^{b}} f \dif x = \lim_{n \to \infty}{\sum_{i = 1}^{k_n}{M_i^{(n)}}}.
\end{equation*}

定义函数列$\{u_n\}, \{U_n\}$为:$u_n(a) = m_1^{(n)}, u_n(x) = m_i^{(n)}, x \in [x_{i - 1}^{(n)}, x_i^{(n)}]$,$U_n(a) = M_1^{(n)}, U_n(x) = M_i^{(n)}, x \in [x_{i - 1}^{(n)}, x_i^{(n)}]$.则它们都是阶梯函数,且$u_n \nearrow, U_n \searrow$,$m \leqslant u_n \leqslant f \leqslant U_n \leqslant M$.令$u = \lim_{n \to \infty}{u_n}$,$U = \lim_{n \to \infty}{U_n}$.则$u, U$为\textbf{有界可测函数},且$u(x) \leqslant f(x) \leqslant U(x), x \in [a, b]$.

\textbf{Step 2:寻找桥梁的性质.}

设$f$是$[a, b]$上的有界可测函数,$\{P_n\}$是$[a, b]$的一列单调加细的分割,且$\lambda_n \to 0$.若$x_0 \in [a, b]$且$x_0$不是任何$\{P_n\}$的分点,则$u(x_0) = U(x_0) \Leftrightarrow f \mbox{在}x_0$处\textbf{连续.}

证明:($\Rightarrow$)设$u(x_0) = U(x_0)$.则$\lim_{n \to \infty} (U_n(x_0) - u_n(x_0)) = U(x_0) - u(x_0) = 0$.$\forall \varepsilon > 0, \exists n_0, s.t. \quad U_{n_0}(x_0) - u_{n_0}(x_0) < \varepsilon$,则在$(x_{i - 1}^{(n_0)}, x_i^{(n_0)})$上$|f(x) - f(x_0)| \leqslant U_{n_0}(x_0) - u_{n_0}(x_0) < \varepsilon$.

($\Leftarrow$)$f$在$x_0$处连续,则$\forall \varepsilon > 0, \exists \delta > 0, s.t. \quad x \in (x_0 - \delta, x_0 + \delta)$,$f(x_0) - \varepsilon < f(x) < f(x_0) + \varepsilon$.取充分大的$n$使得$\lambda_n < \delta$,设$x_0 \in (x_{i - 1}^{(n)}, x_i^{(n)})$,则$[x_{i - 1}^{(n)}, x_i^{(n)}] \subset (x_0 - \delta, x_0 + \delta)$.于是$f(x_0) - \varepsilon \leqslant m_i^{(n)} \leqslant M_i^{(n)} \leqslant f(x_0) + \varepsilon$,即$U_n(x_0) - u_n(x_0) = M_i^{(n)} - m_i^{(n)} \leqslant 2 \varepsilon$. 令$n \to \infty$即得证.

\textbf{Step 3: 找到关系并证明.}

\textbf{设$f$是$[a, b]$上的有界可测函数.}则:
\begin{itemize}[itemindent=2em]
	\item $f \in R([a, b]) \Leftrightarrow f \mbox{在} [a, b] \mbox{上} \mathop{} \! \mathrm{a.e.}$连续.
	\item 若 $f \in R([a, b])$,则$f \in L([a, b])$,且$(R) \int_{a}^{b} f \dif x = (L) \int_{a}^{b} f \dif x$.
\end{itemize}

证明:(1)设$P_n = \{x_0^{(n)}, x_1^{(n)}, \cdots, x_{k_n}^{(n)}\}(n \geqslant 1)$是$[a, b]$上的一列单调可测的分割,且$\lambda_n \to 0$.由有界收敛定理和$u_n, U_n$的关系,有:
\begin{align*}
 	& (L)\int_{a}^{b} U \dif x = \lim_{n \to \infty} (L) \int_{a}^{b} U_n \dif x = \lim_{n \to \infty}{\sum_{i = 1}^{k_n} M_i^{(n)} \Delta x_i^{(n)}}. \\
 	& (L)\int_{a}^{b} u \dif x = \lim_{n \to \infty} (L) \int_{a}^{b} u_n \dif x = \lim_{n \to \infty}{\sum_{i = 1}^{k_n} m_i^{(n)} \Delta x_i^{(n)}}.
\end{align*}

两式相减,可得:
\begin{equation*}
	(L) \int_{a}^{b} (U - u) \dif x = \overline{\int_{a}^{b}} f \dif x - \underline{\int_{a}^{b}} f \dif x.
\end{equation*}

所以$f \in R([a, b]) \Leftrightarrow (L) \int_{a}^{b} (U - u) \dif x = 0 \Leftrightarrow U = u \mathop{} \! \mathrm{a.e.} \! \mathop{} (U - u \geqslant 0)$.

设$A$是分割序列$\{P_n\}$的分点的全体,则$m(A) = 0$.再令$B$是$f$的间断点的全体.$x \notin A, U(x) = u(x) \Leftrightarrow f \mbox{在} x$处连续.从而$f \in R([a, b]) \Leftrightarrow f \mbox{在} [a, b] \mbox{上} \mathop{} \! \mathrm{a.e.}$连续.

(2)首先可得$f = u \mathop{} \! \mathrm{a.e.}$,$f$在$[a, b]$上是可测的.由$f$的有界性可知$f \in L([a, b])$. 故有:
\begin{equation*}
	(L) \int_{a}^{b} f \dif x = (L) \int_{a}^{b} u \dif x = \lim_{n \to \infty}{\sum_{i = 1}^{k_n} m_i^{(n)} \Delta x_i^{(n)}} = (R) \int_{a}^{b} f \dif x.
\end{equation*}

注意:Lebesgue积分的可积函数类\textbf{严格大于}Riemann积分的可积函数类.

\subsection{与广义Riemann积分的关系}
设$\forall b > a$,$f$在$[a, b]$上有界并且几乎处处连续,则$f \in L([a, \infty)) \Leftrightarrow (R) \int_{a}^{b} f \dif x$绝对收敛.且当$(R) \int_{a}^{b} f \dif x$绝对收敛时,有
\begin{equation*}
	(R) \int_{a}^{\infty} f \dif x = (L) \int_{a}^{\infty} f \dif x.
\end{equation*}

证明:$\forall b > a$,$f$在$[a, b]$上有界并且几乎处处连续(正常积分),可知$f \in R([a, b]), f \in L([a, b])$.因此$f$在$[a, \infty)$上是可测的,$\forall n \geqslant a, n \in \mathbb{N}^+$,构造$f$的\textbf{截尾函数列}$f_n(x) = f(x) \chi_{[a, n]}(x)$,则$f$是可测函数列,且$f_n(x) \to f(x)(x \in [a, \infty)), f_n \nearrow$.由Levi定理与正常积分的关系,可知
\begin{align*}
	(R) \int_{a}^{\infty} |f| \dif x & = \lim_{n \to \infty}(R) \int_{a}^{n} |f| \dif x = \lim_{n \to \infty} (L) \int_{a}^{n} |f| \dif x \\
									 & = \lim_{n \to \infty}(L) \int_{[a, \infty)} |f_n| \dif x = (L) \int_{a}^{\infty} |f| \dif x.
\end{align*}

因此$f \in L[a, \infty) \Leftrightarrow (R) \int_{a}^{\infty} f \dif x$绝对收敛.注意到$|f_n| \leqslant |f|(n \geqslant 1)$.利用控制收敛定理与以上的类似方法,得到:
\begin{equation*}
	(R) \int_{a}^{\infty} f \dif x = (L) \int_{a}^{\infty} f \dif x.
\end{equation*}

注意:非绝对收敛的Riemann广义积分不是Lebesgue可积的.如$f(x) = \frac{\sin{x}}{x}$.

$\int_{a}^{b} f \dif x$\textbf{在Riemann正常积分和绝对收敛的广义Riemann积分时与Lebesgue积分等价(注意$f$的有界性).}

\subsection{应用}
\textbf{Lebesgue积分用于理论证明,Riemann积分用于计算.}

应用的两种形式:
\begin{itemize}[itemindent=2em]
	\item 计算积分型函数列的极限,用控制收敛定理即可,重点在于\textbf{寻找一个合适的可测函数,使得能够把原函数控制住,还要Riemann可积}.
	\item 计算原函数无法表示的积分,先用级数展开,然后用逐项积分定理,重点在于\textbf{基本级数展开式的使用与级数收敛域的控制,同时还要构造非负可测的函数列,因为逐项积分定理是由Levi定理推出}.
\end{itemize}

\section{Fubini定理}

\textbf{Fubini定理讨论的是累次积分的换序问题.}

一般形式:若$f(x, y)$是$\mathbb{R}^p \times \mathbb{R}^q$上的非负可测函数则对几乎处处的$x \in \mathbb{R}^p, f(x, y)$作为$y$的函数在$\mathbb{R}^q$上可测,$g(x) = \int_{\mathbb{R}^q f(x, y) \dif y}$在$\mathbb{R}^p$上可测,并且:
\begin{equation*}
	\int_{\mathbb{R}^p \times \mathbb{R}^q} f(x, y) \dif x \dif y = \int_{\mathbb{R}^p}(\int_{\mathbb{R}^q f(x, y)} \dif y) \dif x.
\end{equation*}

将非负可测函数这一条件改为可积函数,得到的可测结论改为可积,上述定理依然成立.

方体形式:$I \subset \mathbb{R}^p, J \subset \mathbb{R}^q$为方体.若$f(x, y)$是$I \times J$上的非负可测函数或可积函数,则:
\begin{equation*}
	\int_{I \times J} f(x, y) \dif x \dif y = \int_I \dif x \int_J f(x, y) \dif y = \int_J \dif y \int_I f(x, y) \dif x. 
\end{equation*}

\textbf{上述定理一般不用在实际计算中,实际计算使用以下的推论.}

$I \subset \mathbb{R}^p, J \subset \mathbb{R}^q$为方体.

\textbf{推论:}$I \subset \mathbb{R}^p, J \subset \mathbb{R}^q$为方体.若$f(x, y)$是$I \times J$上的可测函数,且以下两式中至少有一个成立:
\begin{equation*}
	\int_E \dif x \int_J |f(x, y)| \dif y < \infty, \int_J \dif y \int_I |f(x, y)| \dif x < \infty.
\end{equation*}

则下式成立:
\begin{equation*}
	\int_{I \times J} f(x, y) \dif x \dif y = \int_I \dif x \int_J f(x, y) \dif y = \int_J \dif y \int_I f(x, y) \dif x. 
\end{equation*}

\appendix
\section{布置的课后作业}
\indent $1,2,3,4,6,7,8,9,10,11,12,14,15,16,17,18,19,21,25,27,28,29,30,31,32,33,35.$

\section{课后作业的讲解}
1. $\sum_{i = 1}^{n}{\chi_{A_i}(x)} \geqslant q$,$\int_{0}^{1} \sum_{i = 1}^{n}{\chi_{A_i}(x)} \dif x \geqslant \int_{0}^{1} q \dif x$.

3.$E = E(|f| \geqslant 2) \cup E(|f| \leqslant 2)$.分段处理.注意$\ln{3} > 1$.

4.$E = E(|f| \geqslant \delta) \cup E(|f| \leqslant \delta)$.分段处理.$\delta$由$f'(0)$给出.

6.$E(f > \frac{1}{k}) \nearrow$,有$E(f > 0) = \bigcup_{k = 1}^{\infty}{f \geqslant \frac{1}{k}}$.取测度后取极限可得:
\begin{equation*}
	\lim_{k = \infty} m(E(f \geqslant \frac{1}{k})) = m(\bigcup_{k = 1}^{\infty}{E(f \geqslant \frac{1}{k})}) = m(E) > 0.
\end{equation*}

于是可以得出极限的性质:
\begin{equation*}
	\exists k_0 \mathop{} \! \mathrm{s.t.} \! \mathop{} m(E(f \geqslant \frac{1}{k_0})) > 0.
\end{equation*}

最后再考察积分:
\begin{equation*}
	\int_E f \dif x \geqslant \int_{E(f \geqslant \frac{1}{k_0})} f \dif x \geqslant \int_{E(f \geqslant \frac{1}{k_0})} \frac{1}{k_0} \dif x > 0. 
\end{equation*}

7.即证$m(E(f \neq g)) = 0$.不妨设$f > g$.用第六题结论.

9.利用积分的绝对连续性,注意组织$\varepsilon,\delta$的使用顺序,按极限定义证明即可.反方向使用放缩即可.

11.构造函数,利用介值定理:
\begin{equation*}
	\varphi(x) = \int_{[a, x) \cap E} f \dif x (a \leqslant x < b).
\end{equation*}

12.先可将几乎处处有限设成处处有限的函数.然后记$A_n = m(n \leqslant |f| < n + 1)$,利用:
\begin{equation*}
	\sum_{n = 0}^{\infty}{(n + 1) m(A_n)} = \sum_{n = 0}^{\infty}{n m(A_n)} + m(E) < \infty.
\end{equation*}
\begin{equation*}
	\sum_{n = 0}^{\infty}n m(A_n) \leqslant \sum_{n = 0}^{\infty}{\int_{A_n} |f| \dif x} = \int_E |f| \dif x < \infty.
\end{equation*}

14.令$f_n = f_{\chi_{E_n}} \geqslant 0 \nearrow$,则$f_n \to f (n \to \infty)$.用Levi定理.也可以用积分对积分域的可列可加性.

15.类似第一题的方法:
\begin{equation*}
	\int_E \sum_{n = 1}^{\infty}{\chi_{A_n}(x)} \dif x = \sum_{n = 1}^{\infty}{\int_E \chi_{A_n}(x) \dif x} = \sum_{n = 1}^{\infty}{m(A_n)} < \infty.
\end{equation*}

16.\textbf{注意$\sum_{i = 1}^{\infty}{f_i(x)}$可能无定义,需要先证明它是有定义的.}令$g(x) = \sum_{n = 1}^{\infty}{|f_n(x)|} < \infty$.则有$\int_E g \dif x = \int_E \sum_{n = 1}^{\infty}{|f_n(x)|} \dif x = \sum_{n = 1}^{\infty}{\int_E |f_n| \dif x} < \infty$.则有$\sum_{n = 1}^{\infty}{|f_n(x)|}$a.e.收敛,即$\sum_{n = 1}^{\infty}{f_n(x)}$a.e.收敛.然后使用控制收敛定理.

17.将下极限转化为子列,然后使用Risez定理与Fatou定理.

18.将$g = f - f_n + |f - f_n|$作为控制函数,$0 \leqslant g \leqslant 2f$.证明$\lim_{n \to \infty}{\int_E |f - f_n| \dif x}$.

19.$(\Rightarrow) f_n\stackrel{m}{\longrightarrow} 0 \Rightarrow \frac{|f_n|}{|f_n| + 1} \stackrel{m}{\longrightarrow} 0$.利用测度有限的条件与有限收敛定理.

$(\Leftarrow)$ 利用$\varphi(x) = \frac{x}{1 + x}$的单调性,可得:
\begin{equation*}
	m(E(|f_n| \geqslant \varepsilon)) \leqslant m(E(\frac{|f_n|}{|f_n| + 1} \geqslant \frac{\varepsilon}{1 + \varepsilon})) \leqslant \frac{\varepsilon + 1}{\varepsilon} \int_E \frac{|f_n|}{|f_n| + 1} \dif x.
\end{equation*}

21.归结原理与Lebesgue控制收敛定理.控制函数取为:
\begin{equation*}
	g(x) = 
	\begin{cases}
		1 \quad x \in E(f \leqslant 1) \\
		f(x) \quad x \in E(f > 1).
	\end{cases}
\end{equation*}

25.$f(x)$在$[a, b]$上有界,设$|f(x)| \leqslant M, x \in [a, b]$.因为$g(x)$在$\mathbb{R}^1$上连续.故$g(u)$在$[-M, M]$上有界.设$|g(u)| \leqslant K, u \in [-M, M]$.则当$x \in [a, b], |g(f(x))| \leqslant K$.故$g(f(x))$在$[a, b]$上有界.然后使用Riemann可积的条件.

27.利用积分的单调性与此处的非负性即可.(可以不用控制收敛定理)

29.$f, g$在$[a, b]$上Riemann可积,故存在$[a, b]$的一个零测集$E_0$,使得$f, g \in C([a, b] - E_0)$.由于$f, g$在$[a, b]$的有理数集上相等,$\forall x \in [a, b] - E_0, \exists {x_n} \subset \mathbb{Q} \mathop{} \! s.t. \! \mathop{} x_n \to x$.于是$f(x) = \lim_{n \to \infty}{f(x_n)} = \lim_{n \to \infty}{g(x_n)} = g(x)$.

30.控制函数为:(第一部分用到了均值不等式:$nx \leqslant \frac{1}{2}(1 + n^2 x^2)$)
\begin{equation*}
	g(x) = 
	\begin{cases}
		\frac{\sqrt{x}}{2x} \quad (x \leqslant 1) \\
		\frac{\sqrt{x}}{x^2} \quad (x > 1).
	\end{cases}
\end{equation*}

\section{一点闲谈}
\indent 注意,以下内容仅代表本人观点。

第四章是对Lebesgue积分的基本介绍,不涉及抽象测度空间的内容.

有一些有趣的套路,来跟大家分享一下:
\begin{itemize}[itemindent=2em]
	\item 对于一般函数的证明有一个(非常花时间的)通法:即先说明非负简单函数,再说明非负可测函数,最后说明一般函数.
	\item 将下极限转化为子列非常有用.
	\item 关于控制函数的选择有一些比较好的方法:使用不等式放缩;构造分段函数,在不同的区间上分别控制.
	\item 使用归结原理使得函数极限转为数列极限.然后使用相关定理.
	\item 可以使用特征函数与Lebesgue积分来证明测度的一些问题.
\end{itemize}

还有一些需要注意的点:
\begin{itemize}[itemindent=2em]
	\item 注意\textbf{慎用}小于号,因尽量采用小于等于号.
	\item 注意定理使用的条件,如Riemann可积的大条件中需要被积函数有界.
	\item 注意使用一个公式时,需要验证这个公式是否已经定义.
\end{itemize}


\end{document}
