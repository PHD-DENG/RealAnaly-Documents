\documentclass[bwprint, withoutpreface]{cumcmthesis}

\usepackage{extarrows}
\usepackage{authblk}
\usepackage{tikz}

\title{实变第二章总结}

\begin{document}
\maketitle
\noindent Author: Tony Xiang

\noindent Full Document can be acquired here: 

\noindent https://github.com/T0nyX1ang/RealAnaly-Documents/blob/master/Chapter\%202/Chapter2.pdf

\noindent Full Source code can be downloaded here:

\noindent https://github.com/T0nyX1ang/RealAnaly-Documents/blob/master/Chapter\%202/Chapter2.tex

\section{引言}
\indent 测度的四条性质:\textbf{非负性,可列可加性,平移不变性,可减性}.

区间的长度:
\begin{equation*}
	|I| = 
	\begin{cases}
		b - a, \quad \mbox{$I$为有界区间}, I = (a, b), (a, b], [a, b), [a, b] \\
		+\infty, \quad \mbox{$I$为无界区间}
	\end{cases}	
\end{equation*}

方体的体积:$I = I_1 \times I_2 \times \cdots I_n$,则体积为:$|I| = |I_1| \times |I_2| \times \cdots |I_n|$.

只要有一个区间是无界的,整个方体即为无界方体,其体积为$+\infty$.另外有$|\emptyset| = 0$.

规定:$a + (+\infty) = (+\infty) + a = +\infty$,$(+\infty) + (+\infty) = +\infty$.

\section{外测度}
\indent 开方体覆盖.

\textbf{外测度:}对每个$A \subset \mathbb{R}^n$,令
\begin{equation*}
	m^*(A) = \inf{\sum_{k = 1}^{\infty}\{{I_k}: \{I_k\} \mbox{是$A$的开方体覆盖}\}}.
\end{equation*}
称$m^*(A)$为$A$的Lebesgue外测度.

注意:$\forall A \subset \mathbb{R}^n$,$m^*(A) \geqslant 0$(非负性).若对$A$的任一开方体覆盖$\{I_k\}$,有$\sum_{k = 1}^{\infty}|I_k| = \infty$,则$m^*(A) = \infty$.即$0 \leqslant m^*(A) \leqslant \infty$.

回顾下确界的定义:$a = \inf E$,则
\begin{itemize}[itemindent=2em]
	\item $\forall x \in E, a \leqslant x$.
	\item $\forall \varepsilon > 0, \exists x' \in E, s.t. \quad x' < a + \varepsilon$.
\end{itemize}

由此得出外测度的两个事实:
\begin{itemize}[itemindent=2em]
	\item \textbf{事实1:} $\forall {I_k} \supset A, s.t. \quad m^*(A) \leqslant \sum_{k = 1}^{\infty}|I_k|$.
	\item \textbf{事实2:} $m^*(A) < \infty, \forall \varepsilon > 0, \exists {I_k} \supset A, s.t. \quad \sum_{k = 1}^{\infty}|I_k| < m^*(A) + \varepsilon$.
\end{itemize}

\textbf{一个重要的数分定理:设$a, b \in \mathbb{R}^1, \forall \varepsilon > 0, s.t. \quad a < b + \varepsilon$,则$a \leqslant b$.}

外测度的性质:
\begin{itemize}[itemindent=2em]
	\item 可数集的外测度为$0$.(用$I_k = (a_k - \frac{\varepsilon}{2^{k + 1}}, a_k + \frac{\varepsilon}{2^{k + 1}})$覆盖$A$.)
	\item $m^*(\emptyset) = 0$.
	\item 单调性:$A \subset B \Rightarrow m^*(A) \leqslant m^*(B)$. (由事实1,2与数分定理可得.)
	\item \textbf{次可列可加性:}对$\mathbb{R}^n$中的任意一列集$\{I_k\}$,有:
	\begin{equation*}
		m^*(\bigcup_{k = 1}^{\infty}{I_k}) \leqslant \sum_{k = 1}^{\infty}{m^*(I_k)}.
	\end{equation*}(将事实2进行改进:$\sum_{i = 1}^{\infty}|I_{k, i}| \leqslant m^*(A_k) + \frac{\varepsilon}{2^k}$,之后再二重求和即可.)
	\item \textbf{次有限可加性:}对$\mathbb{R}^n$中的任意一列集$\{I_k\}$,有:
	\begin{equation*}
		m^*(\bigcup_{k = 1}^{n}{I_k}) \leqslant \sum_{k = 1}^{n}{m^*(I_k)}.
	\end{equation*}(次可列可加性的推论.)
	\item 平移不变性:$E \subset \mathbb{R}^n, \forall x_0 \in \mathbb{R}^n, m^*(x_0 + E) = m^*(E)$,$x_0 + E = \{x_0 + x: x \in E\}.$ (由事实1,2与对偶的性质证明.)
	\item 数乘的性质:$E \subset \mathbb{R}^n, \forall \lambda \in \mathbb{R}, m^*(\lambda E) = |\lambda|^n m^*(E)$,$\lambda E = \{\lambda x: x \in E\}.$ (由事实1,2与对偶的性质证明.)
\end{itemize}

注意:次可列可加性是外测度中“不太好”的性质,它没有集合不相交的条件限制.

$m^*(I) = |I|$,$I$为$\mathbb{R}^n$中方体.这是外测度与体积的对应.(分区间是有限的与区间是无限的两个部分证明,有限部分利用事实2,有限覆盖定理,外测度的单调性与夹逼定理证明,无限部分取任意大的数$k$,都有$[a, a + k] \subset [a, +\infty]$,$k \leqslant m^*([a, +\infty])$.)

注意:有限覆盖定理指:一族开区间覆盖一个区间$I$,则必能找到有限的区间,将区间$I$覆盖.

\section{可测集与测度}
\subsection{定义与性质}

\indent $m^*(A) = m^*(A \cap E) + m^*(A \cap E^C) \Leftrightarrow m^*(I) = m^*(I \cap E) + m^*(I \cap E^C)$

\textbf{Caratheodory条件:}$m^*(A) = m^*(A \cap E) + m^*(A \cap E^C)$.

\textbf{可测集:}$E \subset \mathbb{R}^n, \forall A \subset \mathbb{R}^n$,Caratheodory条件成立,称$E$为Lebesgue可测集.

\textbf{可测集的测度:}若$E$为Lebesgue可测集,则称$m^*(E)$为$E$的Lebesgue测度,记为$m(E)$.

Caratheodory条件等价于$m^*(A) \geqslant m^*(A \cap E) + m^*(A \cap E^C)$成立.

\textbf{外测度为零的集合是可测集,称其为零测集.}

\textbf{零测集的子集为可测集.}

\textbf{可数集是可测集,且测度为零,特别的$m(\mathbb{Q}) =0 )$.}

$\mathbb{R}^n$中的每个方体是可测集,且其测度等于方体的体积.(用Caratheodory条件,$J \cap {I^C} = \bigcap_{i = 1}^{k}{I_i}$,且这些集合互不相交.)

可测集的性质:
\begin{itemize}[itemindent=2em]
	\item 若$E_1, E_2, \cdots, E_n$为可测集,则$\bigcup_{i = 1}^{k}{E_k}$为可测集.
	证明在$k = 2$时成立,$E = E_1 \cup E_2$,$E = E_1 \cup (E_2 - E_1) = E_1 \cup ({E_1}^C \cap E_2)$.
	\item 若$E_1, E_2, \cdots, E_n$为互不相交的可测集,$A_i \subset E_i$,则$m^*(\bigcup_{i = 1}^{k}{A_i}) = \sum_{i = 1}^{k}{m^*(A_i)}$.
	证明在$k = 2$时成立,$(A_1 \cup A_2) \cap E_1 = A_1$,$(A_1 \cup A_2) \cap {E_1}^C = A_2$.
	\item 可测集的全体是一个$\sigma$-代数.
	首先可测集的全体是一个代数,证明$\mathcal{M}(\mathbb{R}^n)$对不相交可列并封闭,利用上一个性质,Caratheodory条件并进一步取极限即可.
	\item $\mathcal{B}(\mathbb{R}^n) \subset \mathcal{M}(\mathbb{R}^n)$,且包含关系为严格包含.
	利用开集构造定理与Borel集的定义即可.
	\item \textbf{有限可加性:}$A_1, A_2, \cdots, A_n$为互不相交的可测集,则$m(\bigcup_{i = 1}^{k}{A_i}) = \sum_{i = 1}^{k}{m(A_i)}$.(性质2中取$A_i = E_i$)
	\item \textbf{可减性:}$A, B \in \mathcal{M}(\mathbb{R}^n), A \subset B, m(A) < \infty$,则$m(A - B) = m(A) - m(B).$(注意$A$的测度是有限的,$B = A \cup (B - A)$,$A \cap (B - A) = \emptyset$)
	\item \textbf{可列可加性:}$\{A_k\}$为一列互不相交的可测集,则$m(\bigcup_{k = 1}^{\infty}{A_k}) = \sum_{k = 1}^{\infty}{m(A_k)}$.(取极限,用测度的次可列可加性分别得到正向与反向的不等式)
	\item \textbf{下连续性:}$\{A_k\}$为一列单调递增的可测集,则$m(\bigcup_{k = 1}^{\infty}{A_k}) = \lim_{k \to \infty} m(A_k)$.(构造不交并$B_1 = A_1, B_k = A_k - A_{k - 1}$,测度的可列可加性)
	\item \textbf{上连续性:}$\{A_k\}$为一列单调递减的可测集,且$m(A_1) < \infty$,则$m(\bigcup_{k = 1}^{\infty}{A_k}) = \lim_{k \to \infty} m(A_k)$. (令$B_k = A_1 - A_k$转化为下连续性的问题,注意证明时会使用可减性,所以上连续性与下连续性不完全相同)
	\item 平移不变性:$E \subset \mathbb{R}^n, \forall x_0 \in \mathbb{R}^n, m(x_0 + E) = m(E)$,$x_0 + E = \{x_0 + x: x \in E\}$.(平移不变性由外测度可知,主要证明$x_0 + E \in \mathcal{M}(\mathbb{R}^n)$,使用$x_0 + A \cap E = (x_0 + A) \cap (x_0 + E), x_0 + E^C = (x_0 + E)^C$)
\end{itemize}

注意:测度继承了外测度的所有性质,所以需要对次可列可加性与可列可加性进行区分,可列可加性需要集合相交,次可列可加性不需要.

\textbf{证明一个集合是可测的方法:}
\begin{itemize}[itemindent=2em]
	\item Caratheodory条件.
	\item 证明它是一个Borel集.
	\item 利用可测集的运算间接证明.
\end{itemize}

\subsection{可测集的逼近定理}
\indent 可测集可以用开集,闭集逼近,可以用$G_{\delta},F_{\sigma}$型集最佳逼近.

\begin{itemize}[itemindent=2em]
	\item $\forall \varepsilon > 0, \exists \mbox{开集}G \supset E, s.t. \quad m(G - E) < \varepsilon$.
	\item $\forall \varepsilon > 0, \exists \mbox{闭集}F \subset E, s.t. \quad m(E - F) < \varepsilon$.
	\item $\exists G_{\delta} \mbox{型集} G \supset E, s.t. \quad m(G - E) = 0$.
	\item $\exists F_{\sigma} \mbox{型集} F \subset E, s.t. \quad m(E - F) = 0$.
\end{itemize}

逼近定理的证明思路:
\begin{itemize}[itemindent=2em]
	\item 将$m(E)$分有限与无穷两种情况讨论.对于有限情况,利用外测度的事实2,用一列开方体${I_k}$覆盖之,作$G = \bigcup_{k = 1}^{\infty}{I_k}$,由测度可减性可知.
	\item 对于无限情况,先找一列互不相交的可测集$\{A_k\}$,$m(A_k) < \infty$且$\mathbb{R}^n = \bigcup_{k = 1}^{\infty}{A_k}$,令$E_k = E \cap A_k$,则$m(E_k) < \infty$且$E = \bigcup_{k = 1}^{\infty}{E_k}$,这样将\textbf{无限测度转化至有限情况}.取$G_k \supset E_k, s.t.\quad m(G_k - E_k) < \frac{\varepsilon}{2^k}$,取$G = \bigcup_{k = 1}^{\infty}{G_k}$,$G - E \subset \bigcup_{k = 1}^{\infty}\{G_k - E_k\}$.使用测度单调性与次可列可加性即可.
	\item 对于第二条,将闭集转化至开集的情况.$E - F = E \cap F^C = {(E^C)}^C \cap G = G - E^C, F = G^C$.
	\item 对于第三条,利用$\varepsilon \to \frac{1}{k}$的转化,将极限转化为集列${G_k}$,作$G = \bigcap_{k = 1}^{\infty}{G_k}$,取极限即可,这个方法很重要.
	\item 对于第四条,利用$\varepsilon \to \frac{1}{k}$的转化,将极限转化为集列${F_k}$,作$F = \bigcup_{k = 1}^{\infty}{F_k}$,取极限即可.
\end{itemize}

注意:对偶性可以简化很多计算.

注意:每个可测集与一个Borel集仅相差一个零测集,但是这个差距是集合完备与不完备之间的差距.

$[0, 1]$间存在不可测集,可以使用Zermelo选取公理构造.

\appendix
\section{布置的课后作业}
\indent $1,4,5,6,7,8,9,12,13,15,16,17,18,19,20,21.$

\section{课后作业的讲解}

\end{document}